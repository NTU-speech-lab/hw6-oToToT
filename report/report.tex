\documentclass[a4paper,11pt]{article}
\usepackage[left=1.8cm, right=1.8cm, top=2cm, bottom=2cm]{geometry}
\usepackage{xeCJK}
\usepackage{indentfirst}
\usepackage{tabularx}
\usepackage{graphicx}
\usepackage{amsmath}
\usepackage{listings}
\usepackage{verbatim}
\usepackage{fancyhdr}
% \usepackage[compact]{titlesec}
\usepackage[usenames,dvipsnames]{xcolor}

\setCJKmainfont{NotoSansCJKtc-Thin}
\setmonofont{Consolas}

\definecolor{CodeGreen}{rgb}{0,0.6,0}
\definecolor{CodeGray}{rgb}{0.5,0.5,0.5}
\definecolor{CodeMauve}{rgb}{0.58,0,0.82}
\lstset{
    basicstyle = \ttfamily\footnotesize, 
    breakatwhitespace = false,
    breaklines = true,         
    commentstyle = \color{CodeGreen}\bfseries,
    extendedchars = false,
    keepspaces=true,
    keywordstyle=\color{blue}\bfseries, % keyword style
    language = C++,                     % the language of code
    otherkeywords={string},
    numbers=left,
    numbersep=5pt,
    numberstyle=\tiny\color{CodeGray},
    rulecolor=\color{black},
    showspaces=false,
    showstringspaces=false,
    showtabs=false,
    stepnumber=1,       
    stringstyle=\color{CodeMauve},        % string literal style
    tabsize=2,
}
% from https://blog.csdn.net/RobertChenGuangzhi/article/details/45126785

\title{Machine Learning 2020 - Homework 6 Report}
\author{學號:b08902100, 系級:資工一, 姓名:江昱勳}
\date{}

\begin{document}
\pagestyle{fancy}
\fancyhead[L]{Machine Learning 2020 - Homework 6}
\fancyhead[R]{Author: b08902100 江昱勳}

\maketitle

% \verbatiminput{HW2_S.txt}
% \lstinputlisting{HW2.cpp}

\begin{enumerate}

\item 試說明 hw6\_best.sh 攻擊的方法,包括使用的 proxy model、方法、參數等。此方法和 FGSM 的差異為何?如何影響你的結果?請完整討論。

我採用的攻擊方法是將FGSM稍作修改,執行多次的FGSM,並且加入momentum的參數,讓我們在維持一定的EPS下也能慢慢找出最佳的噪音。而實際上我讓我的程式對於一張圖最多執行50次的FGSM,而若是有先發現已成功攻擊,則會停下來,不繼續執行FGSM;EPS的部份我則是選擇0.02,實際測量過0.01的時候會無法將全部的圖片都成功攻擊,因此我EPS選擇0.02。momentum的參數我則覺得差異不大,有嘗試過設置成0.6也能在差不多的EPS下成功攻擊圖片,而我最後選擇設製成0.8。proxy model的部份稍微實驗過後我採用PyTorch內建的densenet121,理由如第二題。

\item 請嘗試不同的 proxy model,依照你的實作的結果來看,背後的 black box 最有可能為哪一個模型?請說明你的觀察和理由。

我認為背後的 black box model 最有可能是PyTorch內建的densenet121,因為我有嘗試將我的proxy model設置成其他的model,在相同的條件下,成功率都會大幅下降。此外,嘗試將一份稍作攻擊過後的圖片傳至 Judge 上後,可以觀察到densenet121的accuracy與 Judge 上的success rate最為相近,故可以推測背後的model為densenet121。

\item 請以 hw6\_best.sh 的方法,visualize 任意三張圖片攻擊前後的機率圖 (分別取前三高的機率)。

\includegraphics[width=\textwidth]{../plot/123.pdf}
\includegraphics[width=\textwidth]{../plot/124.pdf}
\includegraphics[width=\textwidth]{../plot/125.pdf}

\item 請將你產生出來的 adversarial img,以任一種 smoothing 的方式實作被動防禦 (passive defense),觀察是否有效降低模型的誤判的比例。請說明你的方法,附上你防禦前後的 success rate,並簡要說明你的觀察。另外也請討論此防禦對原始圖片會有什麼影響。

這裡以PyTorch內建的densenet121做測試,使用$2\times 2$的高斯模糊。在未加入高斯模糊前,原始圖片正確率為0.925,被攻擊的圖片正確率則為0。而加入高斯模糊後,原始圖片的正確率降至0.595,但被攻擊的圖片正確率則升回0.52,可以發現兩者差異不大,但是準確率的確被減少了許多。以下附上與第3題相同,但是高斯模糊後的結果。

\includegraphics[width=\textwidth]{../plot/gauss-123.pdf}
\includegraphics[width=\textwidth]{../plot/gauss-124.pdf}
\includegraphics[width=\textwidth]{../plot/gauss-125.pdf}

\end{enumerate}

\end{document}
